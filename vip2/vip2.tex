\documentclass[11pt]{article}

    \usepackage[breakable]{tcolorbox}
    \usepackage{parskip} % Stop auto-indenting (to mimic markdown behaviour)
    
    \usepackage{iftex}
    \ifPDFTeX
    	\usepackage[T1]{fontenc}
    	\usepackage{mathpazo}
    \else
    	\usepackage{fontspec}
    \fi

    % Basic figure setup, for now with no caption control since it's done
    % automatically by Pandoc (which extracts ![](path) syntax from Markdown).
    \usepackage{graphicx}
    % Maintain compatibility with old templates. Remove in nbconvert 6.0
    \let\Oldincludegraphics\includegraphics
    % Ensure that by default, figures have no caption (until we provide a
    % proper Figure object with a Caption API and a way to capture that
    % in the conversion process - todo).
    \usepackage{caption}
    \DeclareCaptionFormat{nocaption}{}
    \captionsetup{format=nocaption,aboveskip=0pt,belowskip=0pt}

    \usepackage{float}
    \floatplacement{figure}{H} % forces figures to be placed at the correct location
    \usepackage{xcolor} % Allow colors to be defined
    \usepackage{enumerate} % Needed for markdown enumerations to work
    \usepackage{geometry} % Used to adjust the document margins
    \usepackage{amsmath} % Equations
    \usepackage{amssymb} % Equations
    \usepackage{textcomp} % defines textquotesingle
    % Hack from http://tex.stackexchange.com/a/47451/13684:
    \AtBeginDocument{%
        \def\PYZsq{\textquotesingle}% Upright quotes in Pygmentized code
    }
    \usepackage{upquote} % Upright quotes for verbatim code
    \usepackage{eurosym} % defines \euro
    \usepackage[mathletters]{ucs} % Extended unicode (utf-8) support
    \usepackage{fancyvrb} % verbatim replacement that allows latex
    \usepackage{grffile} % extends the file name processing of package graphics 
                         % to support a larger range
    \makeatletter % fix for old versions of grffile with XeLaTeX
    \@ifpackagelater{grffile}{2019/11/01}
    {
      % Do nothing on new versions
    }
    {
      \def\Gread@@xetex#1{%
        \IfFileExists{"\Gin@base".bb}%
        {\Gread@eps{\Gin@base.bb}}%
        {\Gread@@xetex@aux#1}%
      }
    }
    \makeatother
    \usepackage[Export]{adjustbox} % Used to constrain images to a maximum size
    \adjustboxset{max size={0.9\linewidth}{0.9\paperheight}}

    % The hyperref package gives us a pdf with properly built
    % internal navigation ('pdf bookmarks' for the table of contents,
    % internal cross-reference links, web links for URLs, etc.)
    \usepackage{hyperref}
    % The default LaTeX title has an obnoxious amount of whitespace. By default,
    % titling removes some of it. It also provides customization options.
    \usepackage{titling}
    \usepackage{longtable} % longtable support required by pandoc >1.10
    \usepackage{booktabs}  % table support for pandoc > 1.12.2
    \usepackage[inline]{enumitem} % IRkernel/repr support (it uses the enumerate* environment)
    \usepackage[normalem]{ulem} % ulem is needed to support strikethroughs (\sout)
                                % normalem makes italics be italics, not underlines
    \usepackage{mathrsfs}
    

    
    % Colors for the hyperref package
    \definecolor{urlcolor}{rgb}{0,.145,.698}
    \definecolor{linkcolor}{rgb}{.71,0.21,0.01}
    \definecolor{citecolor}{rgb}{.12,.54,.11}

    % ANSI colors
    \definecolor{ansi-black}{HTML}{3E424D}
    \definecolor{ansi-black-intense}{HTML}{282C36}
    \definecolor{ansi-red}{HTML}{E75C58}
    \definecolor{ansi-red-intense}{HTML}{B22B31}
    \definecolor{ansi-green}{HTML}{00A250}
    \definecolor{ansi-green-intense}{HTML}{007427}
    \definecolor{ansi-yellow}{HTML}{DDB62B}
    \definecolor{ansi-yellow-intense}{HTML}{B27D12}
    \definecolor{ansi-blue}{HTML}{208FFB}
    \definecolor{ansi-blue-intense}{HTML}{0065CA}
    \definecolor{ansi-magenta}{HTML}{D160C4}
    \definecolor{ansi-magenta-intense}{HTML}{A03196}
    \definecolor{ansi-cyan}{HTML}{60C6C8}
    \definecolor{ansi-cyan-intense}{HTML}{258F8F}
    \definecolor{ansi-white}{HTML}{C5C1B4}
    \definecolor{ansi-white-intense}{HTML}{A1A6B2}
    \definecolor{ansi-default-inverse-fg}{HTML}{FFFFFF}
    \definecolor{ansi-default-inverse-bg}{HTML}{000000}

    % common color for the border for error outputs.
    \definecolor{outerrorbackground}{HTML}{FFDFDF}

    % commands and environments needed by pandoc snippets
    % extracted from the output of `pandoc -s`
    \providecommand{\tightlist}{%
      \setlength{\itemsep}{0pt}\setlength{\parskip}{0pt}}
    \DefineVerbatimEnvironment{Highlighting}{Verbatim}{commandchars=\\\{\}}
    % Add ',fontsize=\small' for more characters per line
    \newenvironment{Shaded}{}{}
    \newcommand{\KeywordTok}[1]{\textcolor[rgb]{0.00,0.44,0.13}{\textbf{{#1}}}}
    \newcommand{\DataTypeTok}[1]{\textcolor[rgb]{0.56,0.13,0.00}{{#1}}}
    \newcommand{\DecValTok}[1]{\textcolor[rgb]{0.25,0.63,0.44}{{#1}}}
    \newcommand{\BaseNTok}[1]{\textcolor[rgb]{0.25,0.63,0.44}{{#1}}}
    \newcommand{\FloatTok}[1]{\textcolor[rgb]{0.25,0.63,0.44}{{#1}}}
    \newcommand{\CharTok}[1]{\textcolor[rgb]{0.25,0.44,0.63}{{#1}}}
    \newcommand{\StringTok}[1]{\textcolor[rgb]{0.25,0.44,0.63}{{#1}}}
    \newcommand{\CommentTok}[1]{\textcolor[rgb]{0.38,0.63,0.69}{\textit{{#1}}}}
    \newcommand{\OtherTok}[1]{\textcolor[rgb]{0.00,0.44,0.13}{{#1}}}
    \newcommand{\AlertTok}[1]{\textcolor[rgb]{1.00,0.00,0.00}{\textbf{{#1}}}}
    \newcommand{\FunctionTok}[1]{\textcolor[rgb]{0.02,0.16,0.49}{{#1}}}
    \newcommand{\RegionMarkerTok}[1]{{#1}}
    \newcommand{\ErrorTok}[1]{\textcolor[rgb]{1.00,0.00,0.00}{\textbf{{#1}}}}
    \newcommand{\NormalTok}[1]{{#1}}
    
    % Additional commands for more recent versions of Pandoc
    \newcommand{\ConstantTok}[1]{\textcolor[rgb]{0.53,0.00,0.00}{{#1}}}
    \newcommand{\SpecialCharTok}[1]{\textcolor[rgb]{0.25,0.44,0.63}{{#1}}}
    \newcommand{\VerbatimStringTok}[1]{\textcolor[rgb]{0.25,0.44,0.63}{{#1}}}
    \newcommand{\SpecialStringTok}[1]{\textcolor[rgb]{0.73,0.40,0.53}{{#1}}}
    \newcommand{\ImportTok}[1]{{#1}}
    \newcommand{\DocumentationTok}[1]{\textcolor[rgb]{0.73,0.13,0.13}{\textit{{#1}}}}
    \newcommand{\AnnotationTok}[1]{\textcolor[rgb]{0.38,0.63,0.69}{\textbf{\textit{{#1}}}}}
    \newcommand{\CommentVarTok}[1]{\textcolor[rgb]{0.38,0.63,0.69}{\textbf{\textit{{#1}}}}}
    \newcommand{\VariableTok}[1]{\textcolor[rgb]{0.10,0.09,0.49}{{#1}}}
    \newcommand{\ControlFlowTok}[1]{\textcolor[rgb]{0.00,0.44,0.13}{\textbf{{#1}}}}
    \newcommand{\OperatorTok}[1]{\textcolor[rgb]{0.40,0.40,0.40}{{#1}}}
    \newcommand{\BuiltInTok}[1]{{#1}}
    \newcommand{\ExtensionTok}[1]{{#1}}
    \newcommand{\PreprocessorTok}[1]{\textcolor[rgb]{0.74,0.48,0.00}{{#1}}}
    \newcommand{\AttributeTok}[1]{\textcolor[rgb]{0.49,0.56,0.16}{{#1}}}
    \newcommand{\InformationTok}[1]{\textcolor[rgb]{0.38,0.63,0.69}{\textbf{\textit{{#1}}}}}
    \newcommand{\WarningTok}[1]{\textcolor[rgb]{0.38,0.63,0.69}{\textbf{\textit{{#1}}}}}
    
    
    % Define a nice break command that doesn't care if a line doesn't already
    % exist.
    \def\br{\hspace*{\fill} \\* }
    % Math Jax compatibility definitions
    \def\gt{>}
    \def\lt{<}
    \let\Oldtex\TeX
    \let\Oldlatex\LaTeX
    \renewcommand{\TeX}{\textrm{\Oldtex}}
    \renewcommand{\LaTeX}{\textrm{\Oldlatex}}
    % Document parameters
    % Document title
    \title{vip2}
    
    
    
    
    
% Pygments definitions
\makeatletter
\def\PY@reset{\let\PY@it=\relax \let\PY@bf=\relax%
    \let\PY@ul=\relax \let\PY@tc=\relax%
    \let\PY@bc=\relax \let\PY@ff=\relax}
\def\PY@tok#1{\csname PY@tok@#1\endcsname}
\def\PY@toks#1+{\ifx\relax#1\empty\else%
    \PY@tok{#1}\expandafter\PY@toks\fi}
\def\PY@do#1{\PY@bc{\PY@tc{\PY@ul{%
    \PY@it{\PY@bf{\PY@ff{#1}}}}}}}
\def\PY#1#2{\PY@reset\PY@toks#1+\relax+\PY@do{#2}}

\@namedef{PY@tok@w}{\def\PY@tc##1{\textcolor[rgb]{0.73,0.73,0.73}{##1}}}
\@namedef{PY@tok@c}{\let\PY@it=\textit\def\PY@tc##1{\textcolor[rgb]{0.25,0.50,0.50}{##1}}}
\@namedef{PY@tok@cp}{\def\PY@tc##1{\textcolor[rgb]{0.74,0.48,0.00}{##1}}}
\@namedef{PY@tok@k}{\let\PY@bf=\textbf\def\PY@tc##1{\textcolor[rgb]{0.00,0.50,0.00}{##1}}}
\@namedef{PY@tok@kp}{\def\PY@tc##1{\textcolor[rgb]{0.00,0.50,0.00}{##1}}}
\@namedef{PY@tok@kt}{\def\PY@tc##1{\textcolor[rgb]{0.69,0.00,0.25}{##1}}}
\@namedef{PY@tok@o}{\def\PY@tc##1{\textcolor[rgb]{0.40,0.40,0.40}{##1}}}
\@namedef{PY@tok@ow}{\let\PY@bf=\textbf\def\PY@tc##1{\textcolor[rgb]{0.67,0.13,1.00}{##1}}}
\@namedef{PY@tok@nb}{\def\PY@tc##1{\textcolor[rgb]{0.00,0.50,0.00}{##1}}}
\@namedef{PY@tok@nf}{\def\PY@tc##1{\textcolor[rgb]{0.00,0.00,1.00}{##1}}}
\@namedef{PY@tok@nc}{\let\PY@bf=\textbf\def\PY@tc##1{\textcolor[rgb]{0.00,0.00,1.00}{##1}}}
\@namedef{PY@tok@nn}{\let\PY@bf=\textbf\def\PY@tc##1{\textcolor[rgb]{0.00,0.00,1.00}{##1}}}
\@namedef{PY@tok@ne}{\let\PY@bf=\textbf\def\PY@tc##1{\textcolor[rgb]{0.82,0.25,0.23}{##1}}}
\@namedef{PY@tok@nv}{\def\PY@tc##1{\textcolor[rgb]{0.10,0.09,0.49}{##1}}}
\@namedef{PY@tok@no}{\def\PY@tc##1{\textcolor[rgb]{0.53,0.00,0.00}{##1}}}
\@namedef{PY@tok@nl}{\def\PY@tc##1{\textcolor[rgb]{0.63,0.63,0.00}{##1}}}
\@namedef{PY@tok@ni}{\let\PY@bf=\textbf\def\PY@tc##1{\textcolor[rgb]{0.60,0.60,0.60}{##1}}}
\@namedef{PY@tok@na}{\def\PY@tc##1{\textcolor[rgb]{0.49,0.56,0.16}{##1}}}
\@namedef{PY@tok@nt}{\let\PY@bf=\textbf\def\PY@tc##1{\textcolor[rgb]{0.00,0.50,0.00}{##1}}}
\@namedef{PY@tok@nd}{\def\PY@tc##1{\textcolor[rgb]{0.67,0.13,1.00}{##1}}}
\@namedef{PY@tok@s}{\def\PY@tc##1{\textcolor[rgb]{0.73,0.13,0.13}{##1}}}
\@namedef{PY@tok@sd}{\let\PY@it=\textit\def\PY@tc##1{\textcolor[rgb]{0.73,0.13,0.13}{##1}}}
\@namedef{PY@tok@si}{\let\PY@bf=\textbf\def\PY@tc##1{\textcolor[rgb]{0.73,0.40,0.53}{##1}}}
\@namedef{PY@tok@se}{\let\PY@bf=\textbf\def\PY@tc##1{\textcolor[rgb]{0.73,0.40,0.13}{##1}}}
\@namedef{PY@tok@sr}{\def\PY@tc##1{\textcolor[rgb]{0.73,0.40,0.53}{##1}}}
\@namedef{PY@tok@ss}{\def\PY@tc##1{\textcolor[rgb]{0.10,0.09,0.49}{##1}}}
\@namedef{PY@tok@sx}{\def\PY@tc##1{\textcolor[rgb]{0.00,0.50,0.00}{##1}}}
\@namedef{PY@tok@m}{\def\PY@tc##1{\textcolor[rgb]{0.40,0.40,0.40}{##1}}}
\@namedef{PY@tok@gh}{\let\PY@bf=\textbf\def\PY@tc##1{\textcolor[rgb]{0.00,0.00,0.50}{##1}}}
\@namedef{PY@tok@gu}{\let\PY@bf=\textbf\def\PY@tc##1{\textcolor[rgb]{0.50,0.00,0.50}{##1}}}
\@namedef{PY@tok@gd}{\def\PY@tc##1{\textcolor[rgb]{0.63,0.00,0.00}{##1}}}
\@namedef{PY@tok@gi}{\def\PY@tc##1{\textcolor[rgb]{0.00,0.63,0.00}{##1}}}
\@namedef{PY@tok@gr}{\def\PY@tc##1{\textcolor[rgb]{1.00,0.00,0.00}{##1}}}
\@namedef{PY@tok@ge}{\let\PY@it=\textit}
\@namedef{PY@tok@gs}{\let\PY@bf=\textbf}
\@namedef{PY@tok@gp}{\let\PY@bf=\textbf\def\PY@tc##1{\textcolor[rgb]{0.00,0.00,0.50}{##1}}}
\@namedef{PY@tok@go}{\def\PY@tc##1{\textcolor[rgb]{0.53,0.53,0.53}{##1}}}
\@namedef{PY@tok@gt}{\def\PY@tc##1{\textcolor[rgb]{0.00,0.27,0.87}{##1}}}
\@namedef{PY@tok@err}{\def\PY@bc##1{{\setlength{\fboxsep}{\string -\fboxrule}\fcolorbox[rgb]{1.00,0.00,0.00}{1,1,1}{\strut ##1}}}}
\@namedef{PY@tok@kc}{\let\PY@bf=\textbf\def\PY@tc##1{\textcolor[rgb]{0.00,0.50,0.00}{##1}}}
\@namedef{PY@tok@kd}{\let\PY@bf=\textbf\def\PY@tc##1{\textcolor[rgb]{0.00,0.50,0.00}{##1}}}
\@namedef{PY@tok@kn}{\let\PY@bf=\textbf\def\PY@tc##1{\textcolor[rgb]{0.00,0.50,0.00}{##1}}}
\@namedef{PY@tok@kr}{\let\PY@bf=\textbf\def\PY@tc##1{\textcolor[rgb]{0.00,0.50,0.00}{##1}}}
\@namedef{PY@tok@bp}{\def\PY@tc##1{\textcolor[rgb]{0.00,0.50,0.00}{##1}}}
\@namedef{PY@tok@fm}{\def\PY@tc##1{\textcolor[rgb]{0.00,0.00,1.00}{##1}}}
\@namedef{PY@tok@vc}{\def\PY@tc##1{\textcolor[rgb]{0.10,0.09,0.49}{##1}}}
\@namedef{PY@tok@vg}{\def\PY@tc##1{\textcolor[rgb]{0.10,0.09,0.49}{##1}}}
\@namedef{PY@tok@vi}{\def\PY@tc##1{\textcolor[rgb]{0.10,0.09,0.49}{##1}}}
\@namedef{PY@tok@vm}{\def\PY@tc##1{\textcolor[rgb]{0.10,0.09,0.49}{##1}}}
\@namedef{PY@tok@sa}{\def\PY@tc##1{\textcolor[rgb]{0.73,0.13,0.13}{##1}}}
\@namedef{PY@tok@sb}{\def\PY@tc##1{\textcolor[rgb]{0.73,0.13,0.13}{##1}}}
\@namedef{PY@tok@sc}{\def\PY@tc##1{\textcolor[rgb]{0.73,0.13,0.13}{##1}}}
\@namedef{PY@tok@dl}{\def\PY@tc##1{\textcolor[rgb]{0.73,0.13,0.13}{##1}}}
\@namedef{PY@tok@s2}{\def\PY@tc##1{\textcolor[rgb]{0.73,0.13,0.13}{##1}}}
\@namedef{PY@tok@sh}{\def\PY@tc##1{\textcolor[rgb]{0.73,0.13,0.13}{##1}}}
\@namedef{PY@tok@s1}{\def\PY@tc##1{\textcolor[rgb]{0.73,0.13,0.13}{##1}}}
\@namedef{PY@tok@mb}{\def\PY@tc##1{\textcolor[rgb]{0.40,0.40,0.40}{##1}}}
\@namedef{PY@tok@mf}{\def\PY@tc##1{\textcolor[rgb]{0.40,0.40,0.40}{##1}}}
\@namedef{PY@tok@mh}{\def\PY@tc##1{\textcolor[rgb]{0.40,0.40,0.40}{##1}}}
\@namedef{PY@tok@mi}{\def\PY@tc##1{\textcolor[rgb]{0.40,0.40,0.40}{##1}}}
\@namedef{PY@tok@il}{\def\PY@tc##1{\textcolor[rgb]{0.40,0.40,0.40}{##1}}}
\@namedef{PY@tok@mo}{\def\PY@tc##1{\textcolor[rgb]{0.40,0.40,0.40}{##1}}}
\@namedef{PY@tok@ch}{\let\PY@it=\textit\def\PY@tc##1{\textcolor[rgb]{0.25,0.50,0.50}{##1}}}
\@namedef{PY@tok@cm}{\let\PY@it=\textit\def\PY@tc##1{\textcolor[rgb]{0.25,0.50,0.50}{##1}}}
\@namedef{PY@tok@cpf}{\let\PY@it=\textit\def\PY@tc##1{\textcolor[rgb]{0.25,0.50,0.50}{##1}}}
\@namedef{PY@tok@c1}{\let\PY@it=\textit\def\PY@tc##1{\textcolor[rgb]{0.25,0.50,0.50}{##1}}}
\@namedef{PY@tok@cs}{\let\PY@it=\textit\def\PY@tc##1{\textcolor[rgb]{0.25,0.50,0.50}{##1}}}

\def\PYZbs{\char`\\}
\def\PYZus{\char`\_}
\def\PYZob{\char`\{}
\def\PYZcb{\char`\}}
\def\PYZca{\char`\^}
\def\PYZam{\char`\&}
\def\PYZlt{\char`\<}
\def\PYZgt{\char`\>}
\def\PYZsh{\char`\#}
\def\PYZpc{\char`\%}
\def\PYZdl{\char`\$}
\def\PYZhy{\char`\-}
\def\PYZsq{\char`\'}
\def\PYZdq{\char`\"}
\def\PYZti{\char`\~}
% for compatibility with earlier versions
\def\PYZat{@}
\def\PYZlb{[}
\def\PYZrb{]}
\makeatother


    % For linebreaks inside Verbatim environment from package fancyvrb. 
    \makeatletter
        \newbox\Wrappedcontinuationbox 
        \newbox\Wrappedvisiblespacebox 
        \newcommand*\Wrappedvisiblespace {\textcolor{red}{\textvisiblespace}} 
        \newcommand*\Wrappedcontinuationsymbol {\textcolor{red}{\llap{\tiny$\m@th\hookrightarrow$}}} 
        \newcommand*\Wrappedcontinuationindent {3ex } 
        \newcommand*\Wrappedafterbreak {\kern\Wrappedcontinuationindent\copy\Wrappedcontinuationbox} 
        % Take advantage of the already applied Pygments mark-up to insert 
        % potential linebreaks for TeX processing. 
        %        {, <, #, %, $, ' and ": go to next line. 
        %        _, }, ^, &, >, - and ~: stay at end of broken line. 
        % Use of \textquotesingle for straight quote. 
        \newcommand*\Wrappedbreaksatspecials {% 
            \def\PYGZus{\discretionary{\char`\_}{\Wrappedafterbreak}{\char`\_}}% 
            \def\PYGZob{\discretionary{}{\Wrappedafterbreak\char`\{}{\char`\{}}% 
            \def\PYGZcb{\discretionary{\char`\}}{\Wrappedafterbreak}{\char`\}}}% 
            \def\PYGZca{\discretionary{\char`\^}{\Wrappedafterbreak}{\char`\^}}% 
            \def\PYGZam{\discretionary{\char`\&}{\Wrappedafterbreak}{\char`\&}}% 
            \def\PYGZlt{\discretionary{}{\Wrappedafterbreak\char`\<}{\char`\<}}% 
            \def\PYGZgt{\discretionary{\char`\>}{\Wrappedafterbreak}{\char`\>}}% 
            \def\PYGZsh{\discretionary{}{\Wrappedafterbreak\char`\#}{\char`\#}}% 
            \def\PYGZpc{\discretionary{}{\Wrappedafterbreak\char`\%}{\char`\%}}% 
            \def\PYGZdl{\discretionary{}{\Wrappedafterbreak\char`\$}{\char`\$}}% 
            \def\PYGZhy{\discretionary{\char`\-}{\Wrappedafterbreak}{\char`\-}}% 
            \def\PYGZsq{\discretionary{}{\Wrappedafterbreak\textquotesingle}{\textquotesingle}}% 
            \def\PYGZdq{\discretionary{}{\Wrappedafterbreak\char`\"}{\char`\"}}% 
            \def\PYGZti{\discretionary{\char`\~}{\Wrappedafterbreak}{\char`\~}}% 
        } 
        % Some characters . , ; ? ! / are not pygmentized. 
        % This macro makes them "active" and they will insert potential linebreaks 
        \newcommand*\Wrappedbreaksatpunct {% 
            \lccode`\~`\.\lowercase{\def~}{\discretionary{\hbox{\char`\.}}{\Wrappedafterbreak}{\hbox{\char`\.}}}% 
            \lccode`\~`\,\lowercase{\def~}{\discretionary{\hbox{\char`\,}}{\Wrappedafterbreak}{\hbox{\char`\,}}}% 
            \lccode`\~`\;\lowercase{\def~}{\discretionary{\hbox{\char`\;}}{\Wrappedafterbreak}{\hbox{\char`\;}}}% 
            \lccode`\~`\:\lowercase{\def~}{\discretionary{\hbox{\char`\:}}{\Wrappedafterbreak}{\hbox{\char`\:}}}% 
            \lccode`\~`\?\lowercase{\def~}{\discretionary{\hbox{\char`\?}}{\Wrappedafterbreak}{\hbox{\char`\?}}}% 
            \lccode`\~`\!\lowercase{\def~}{\discretionary{\hbox{\char`\!}}{\Wrappedafterbreak}{\hbox{\char`\!}}}% 
            \lccode`\~`\/\lowercase{\def~}{\discretionary{\hbox{\char`\/}}{\Wrappedafterbreak}{\hbox{\char`\/}}}% 
            \catcode`\.\active
            \catcode`\,\active 
            \catcode`\;\active
            \catcode`\:\active
            \catcode`\?\active
            \catcode`\!\active
            \catcode`\/\active 
            \lccode`\~`\~ 	
        }
    \makeatother

    \let\OriginalVerbatim=\Verbatim
    \makeatletter
    \renewcommand{\Verbatim}[1][1]{%
        %\parskip\z@skip
        \sbox\Wrappedcontinuationbox {\Wrappedcontinuationsymbol}%
        \sbox\Wrappedvisiblespacebox {\FV@SetupFont\Wrappedvisiblespace}%
        \def\FancyVerbFormatLine ##1{\hsize\linewidth
            \vtop{\raggedright\hyphenpenalty\z@\exhyphenpenalty\z@
                \doublehyphendemerits\z@\finalhyphendemerits\z@
                \strut ##1\strut}%
        }%
        % If the linebreak is at a space, the latter will be displayed as visible
        % space at end of first line, and a continuation symbol starts next line.
        % Stretch/shrink are however usually zero for typewriter font.
        \def\FV@Space {%
            \nobreak\hskip\z@ plus\fontdimen3\font minus\fontdimen4\font
            \discretionary{\copy\Wrappedvisiblespacebox}{\Wrappedafterbreak}
            {\kern\fontdimen2\font}%
        }%
        
        % Allow breaks at special characters using \PYG... macros.
        \Wrappedbreaksatspecials
        % Breaks at punctuation characters . , ; ? ! and / need catcode=\active 	
        \OriginalVerbatim[#1,codes*=\Wrappedbreaksatpunct]%
    }
    \makeatother

    % Exact colors from NB
    \definecolor{incolor}{HTML}{303F9F}
    \definecolor{outcolor}{HTML}{D84315}
    \definecolor{cellborder}{HTML}{CFCFCF}
    \definecolor{cellbackground}{HTML}{F7F7F7}
    
    % prompt
    \makeatletter
    \newcommand{\boxspacing}{\kern\kvtcb@left@rule\kern\kvtcb@boxsep}
    \makeatother
    \newcommand{\prompt}[4]{
        {\ttfamily\llap{{\color{#2}[#3]:\hspace{3pt}#4}}\vspace{-\baselineskip}}
    }
    

    
    % Prevent overflowing lines due to hard-to-break entities
    \sloppy 
    % Setup hyperref package
    \hypersetup{
      breaklinks=true,  % so long urls are correctly broken across lines
      colorlinks=true,
      urlcolor=urlcolor,
      linkcolor=linkcolor,
      citecolor=citecolor,
      }
    % Slightly bigger margins than the latex defaults
    
    \geometry{verbose,tmargin=1in,bmargin=1in,lmargin=1in,rmargin=1in}
    
    

\begin{document}
    
    \maketitle
    
    

    
    \hypertarget{vision-and-imagine-processing-assignment-2}{%
\subsubsection{Vision and Imagine Processing Assignment
2}\label{vision-and-imagine-processing-assignment-2}}

Tibor Krols, Marie Mortensen, Rachel Rea, Ninell Oldenburg

    \begin{tcolorbox}[breakable, size=fbox, boxrule=1pt, pad at break*=1mm,colback=cellbackground, colframe=cellborder]
\prompt{In}{incolor}{1}{\boxspacing}
\begin{Verbatim}[commandchars=\\\{\}]
\PY{k+kn}{import} \PY{n+nn}{numpy} \PY{k}{as} \PY{n+nn}{np}
\PY{k+kn}{import} \PY{n+nn}{matplotlib}\PY{n+nn}{.}\PY{n+nn}{pyplot} \PY{k}{as} \PY{n+nn}{plt}
\PY{n}{plt}\PY{o}{.}\PY{n}{rcParams}\PY{p}{[}\PY{l+s+s1}{\PYZsq{}}\PY{l+s+s1}{figure.figsize}\PY{l+s+s1}{\PYZsq{}}\PY{p}{]} \PY{o}{=} \PY{p}{[}\PY{l+m+mi}{12}\PY{p}{,} \PY{l+m+mi}{12}\PY{p}{]}
\PY{k+kn}{from} \PY{n+nn}{skimage}\PY{n+nn}{.}\PY{n+nn}{data} \PY{k+kn}{import} \PY{n}{camera}
\PY{k+kn}{from} \PY{n+nn}{skimage}\PY{n+nn}{.}\PY{n+nn}{filters} \PY{k+kn}{import} \PY{n}{gaussian}
\PY{c+c1}{\PYZsh{}from skimage.filters.rank import mean}
\PY{k+kn}{from} \PY{n+nn}{skimage}\PY{n+nn}{.}\PY{n+nn}{io} \PY{k+kn}{import} \PY{n}{imread}
\PY{k+kn}{from} \PY{n+nn}{skimage} \PY{k+kn}{import} \PY{n}{img\PYZus{}as\PYZus{}float}
\PY{k+kn}{import} \PY{n+nn}{cv2}
\PY{c+c1}{\PYZsh{}from skimage.feature import canny}
\PY{c+c1}{\PYZsh{}from skimage.util import random\PYZus{}noise}
\PY{c+c1}{\PYZsh{}from skimage.morphology import disk}
\PY{c+c1}{\PYZsh{}from scipy.signal import correlate2d}
\PY{k+kn}{from} \PY{n+nn}{scipy}\PY{n+nn}{.}\PY{n+nn}{ndimage} \PY{k+kn}{import} \PY{n}{gaussian\PYZus{}filter}\PY{p}{,} \PY{n}{gaussian\PYZus{}gradient\PYZus{}magnitude}
\PY{c+c1}{\PYZsh{}from scalespace import gaussian\PYZus{}scalespace, structure\PYZus{}tensor, determinant\PYZus{}symmetric\PYZus{}field, trace\PYZus{}symmetric\PYZus{}field}
\end{Verbatim}
\end{tcolorbox}

    \begin{tcolorbox}[breakable, size=fbox, boxrule=1pt, pad at break*=1mm,colback=cellbackground, colframe=cellborder]
\prompt{In}{incolor}{2}{\boxspacing}
\begin{Verbatim}[commandchars=\\\{\}]
\PY{n}{l} \PY{o}{=} \PY{n}{np}\PY{o}{.}\PY{n}{array}\PY{p}{(}\PY{n}{imread}\PY{p}{(}\PY{l+s+s2}{\PYZdq{}}\PY{l+s+s2}{lenna.jpg}\PY{l+s+s2}{\PYZdq{}}\PY{p}{)}\PY{p}{)}
\PY{n}{l} \PY{o}{=} \PY{n}{img\PYZus{}as\PYZus{}float}\PY{p}{(}\PY{n}{l}\PY{p}{)}
\PY{n}{plt}\PY{o}{.}\PY{n}{imshow}\PY{p}{(}\PY{n}{l}\PY{p}{,} \PY{n}{cmap}\PY{o}{=}\PY{l+s+s1}{\PYZsq{}}\PY{l+s+s1}{Greys\PYZus{}r}\PY{l+s+s1}{\PYZsq{}}\PY{p}{)}
\PY{n}{plt}\PY{o}{.}\PY{n}{gcf}\PY{p}{(}\PY{p}{)}
\PY{n}{plt}\PY{o}{.}\PY{n}{axis}\PY{p}{(}\PY{l+s+s1}{\PYZsq{}}\PY{l+s+s1}{off}\PY{l+s+s1}{\PYZsq{}}\PY{p}{)}
\PY{n}{plt}\PY{o}{.}\PY{n}{show}\PY{p}{(}\PY{p}{)}
\end{Verbatim}
\end{tcolorbox}

    \begin{center}
    \adjustimage{max size={0.9\linewidth}{0.9\paperheight}}{output_2_0.png}
    \end{center}
    { \hspace*{\fill} \\}
    
    \hypertarget{gaussian-filtering}{%
\section{Gaussian Filtering}\label{gaussian-filtering}}

    \hypertarget{first-diy-attempt}{%
\subsection{First DIY attempt}\label{first-diy-attempt}}

    First, we tried to create a Gaussian function and a convolution
ourselves. Therefore, we started creating a gaussian distributed matrix
with the \texttt{random.normal} package. The convolution takes the image
and the matrix as input and convolutes (multiplies and adds) the image
values by the matrix values to a new image.

We can see, that depending on the size of the kernel, the image gets
more and more ``layers''. It looks like more and more thin layers of the
pictures overlap. That makes sense: the bigger kernel is, the more
values are the input of the resulting value in the new picture.
Secondly, of course the edge gets larger as bigger the kernel gets. This
is due to our poor implementation of edge cutting.

Furthermore, we tried varying the sigma value, so the variance of the
gaussian function. The larger the sigma (and closer to the size of the
kernel), the more uniform the gaussian values get. That means, that also
all the values that are being convulted with that kernel are getting
more uniform. Something that we can see clearly in the pictures below.
Where sigma = 1 still preserves most of the color differences, the
larger sigma gets, the less color difference can we see in the picture.

    \begin{tcolorbox}[breakable, size=fbox, boxrule=1pt, pad at break*=1mm,colback=cellbackground, colframe=cellborder]
\prompt{In}{incolor}{8}{\boxspacing}
\begin{Verbatim}[commandchars=\\\{\}]
\PY{k+kn}{import} \PY{n+nn}{random}

\PY{c+c1}{\PYZsh{} create a matrix with gauss distributed values over specified size and sigma}
\PY{k}{def} \PY{n+nf}{gaus\PYZus{}matr}\PY{p}{(}\PY{n}{size}\PY{p}{,} \PY{n}{sigma}\PY{p}{)}\PY{p}{:}
    \PY{k}{return} \PY{n}{np}\PY{o}{.}\PY{n}{random}\PY{o}{.}\PY{n}{normal}\PY{p}{(}\PY{l+m+mi}{0}\PY{p}{,} \PY{n}{sigma}\PY{p}{,} \PY{p}{(}\PY{n}{size}\PY{p}{,} \PY{n}{size}\PY{p}{)}\PY{p}{)}

\PY{k}{def} \PY{n+nf}{convolution}\PY{p}{(}\PY{n}{img}\PY{p}{,} \PY{n}{conv\PYZus{}matrix}\PY{p}{)}\PY{p}{:}
    \PY{c+c1}{\PYZsh{} apply cross correlation and initialize output matrix with same size as original image}
    \PY{n}{conv\PYZus{}matrix} \PY{o}{=} \PY{n}{np}\PY{o}{.}\PY{n}{fliplr}\PY{p}{(}\PY{n}{np}\PY{o}{.}\PY{n}{flipud}\PY{p}{(}\PY{n}{conv\PYZus{}matrix}\PY{p}{)}\PY{p}{)}
    \PY{n}{output} \PY{o}{=} \PY{n}{np}\PY{o}{.}\PY{n}{zeros}\PY{p}{(}\PY{p}{(}\PY{n}{img}\PY{o}{.}\PY{n}{shape}\PY{p}{[}\PY{l+m+mi}{0}\PY{p}{]}\PY{p}{,} \PY{n}{img}\PY{o}{.}\PY{n}{shape}\PY{p}{[}\PY{l+m+mi}{1}\PY{p}{]}\PY{p}{)}\PY{p}{)}
    \PY{c+c1}{\PYZsh{} fill matrix with convoluted values}
    \PY{c+c1}{\PYZsh{} ignore edges at this step}
    \PY{k}{for} \PY{n}{i} \PY{o+ow}{in} \PY{n+nb}{range}\PY{p}{(}\PY{n}{img}\PY{o}{.}\PY{n}{shape}\PY{p}{[}\PY{l+m+mi}{0}\PY{p}{]}\PY{p}{)}\PY{p}{:}
        \PY{k}{for} \PY{n}{j} \PY{o+ow}{in} \PY{n+nb}{range}\PY{p}{(}\PY{n}{img}\PY{o}{.}\PY{n}{shape}\PY{p}{[}\PY{l+m+mi}{1}\PY{p}{]}\PY{p}{)}\PY{p}{:}
            \PY{k}{if} \PY{o+ow}{not} \PY{n}{i}\PY{o}{+}\PY{n}{conv\PYZus{}matrix}\PY{o}{.}\PY{n}{shape}\PY{p}{[}\PY{l+m+mi}{0}\PY{p}{]} \PY{o}{\PYZgt{}} \PY{n}{output}\PY{o}{.}\PY{n}{shape}\PY{p}{[}\PY{l+m+mi}{0}\PY{p}{]} \PY{o+ow}{and} \PY{o+ow}{not} \PY{n}{j}\PY{o}{+}\PY{n}{conv\PYZus{}matrix}\PY{o}{.}\PY{n}{shape}\PY{p}{[}\PY{l+m+mi}{1}\PY{p}{]} \PY{o}{\PYZgt{}} \PY{n}{output}\PY{o}{.}\PY{n}{shape}\PY{p}{[}\PY{l+m+mi}{1}\PY{p}{]}\PY{p}{:}
                \PY{n}{output}\PY{p}{[}\PY{n}{i}\PY{p}{,} \PY{n}{j}\PY{p}{]} \PY{o}{=} \PY{p}{(}\PY{n}{conv\PYZus{}matrix}\PY{o}{*}\PY{n}{img}\PY{p}{[}\PY{n}{i}\PY{p}{:}\PY{n}{i}\PY{o}{+}\PY{n}{conv\PYZus{}matrix}\PY{o}{.}\PY{n}{shape}\PY{p}{[}\PY{l+m+mi}{0}\PY{p}{]}\PY{p}{,} \PY{n}{j}\PY{p}{:}\PY{n}{j}\PY{o}{+}\PY{n}{conv\PYZus{}matrix}\PY{o}{.}\PY{n}{shape}\PY{p}{[}\PY{l+m+mi}{1}\PY{p}{]}\PY{p}{]}\PY{p}{)}\PY{o}{.}\PY{n}{sum}\PY{p}{(}\PY{p}{)}
            
    \PY{k}{return} \PY{n}{output}
\end{Verbatim}
\end{tcolorbox}

    \begin{tcolorbox}[breakable, size=fbox, boxrule=1pt, pad at break*=1mm,colback=cellbackground, colframe=cellborder]
\prompt{In}{incolor}{9}{\boxspacing}
\begin{Verbatim}[commandchars=\\\{\}]
\PY{c+c1}{\PYZsh{} the bigger the size of the kernel, the more blury and all\PYZhy{}grey the image gets}
\PY{n}{fig}\PY{p}{,} \PY{n}{axes} \PY{o}{=} \PY{n}{plt}\PY{o}{.}\PY{n}{subplots}\PY{p}{(}\PY{l+m+mi}{2}\PY{p}{,} \PY{l+m+mi}{2}\PY{p}{)}

\PY{k}{for} \PY{n}{i} \PY{o+ow}{in} \PY{n+nb}{range}\PY{p}{(}\PY{l+m+mi}{4}\PY{p}{)}\PY{p}{:}
    \PY{n}{con\PYZus{}matr} \PY{o}{=} \PY{n}{gaus\PYZus{}matr}\PY{p}{(}\PY{l+m+mi}{4}\PY{o}{*}\PY{o}{*}\PY{n}{i}\PY{p}{,} \PY{l+m+mi}{10}\PY{p}{)}
    \PY{n}{fglenna} \PY{o}{=} \PY{n}{convolution}\PY{p}{(}\PY{n}{l}\PY{p}{,} \PY{n}{con\PYZus{}matr}\PY{p}{)}
    \PY{n}{ax} \PY{o}{=} \PY{n}{axes}\PY{p}{[}\PY{n}{i}\PY{o}{/}\PY{o}{/}\PY{l+m+mi}{2}\PY{p}{,} \PY{n}{i}\PY{o}{\PYZpc{}}\PY{k}{2}]    
    \PY{n}{ax}\PY{o}{.}\PY{n}{imshow}\PY{p}{(}\PY{n}{fglenna}\PY{p}{,}\PY{n}{cmap}\PY{o}{=}\PY{l+s+s1}{\PYZsq{}}\PY{l+s+s1}{Greys\PYZus{}r}\PY{l+s+s1}{\PYZsq{}}\PY{p}{,}\PY{n}{interpolation}\PY{o}{=}\PY{l+s+s1}{\PYZsq{}}\PY{l+s+s1}{none}\PY{l+s+s1}{\PYZsq{}}\PY{p}{)}
    \PY{n}{ax}\PY{o}{.}\PY{n}{set\PYZus{}title}\PY{p}{(}\PY{l+s+s2}{\PYZdq{}}\PY{l+s+s2}{Size of kernel = }\PY{l+s+si}{\PYZob{}\PYZcb{}}\PY{l+s+s2}{\PYZdq{}}\PY{o}{.}\PY{n}{format}\PY{p}{(}\PY{l+m+mi}{4}\PY{o}{*}\PY{o}{*}\PY{n}{i}\PY{p}{)}\PY{p}{)}
\PY{n}{fig}\PY{o}{.}\PY{n}{suptitle}\PY{p}{(}\PY{l+s+s1}{\PYZsq{}}\PY{l+s+s1}{Gaussian Filter with different kernel sizes, sigma = 10}\PY{l+s+s1}{\PYZsq{}}\PY{p}{)}\PY{p}{;}
\end{Verbatim}
\end{tcolorbox}

    \begin{center}
    \adjustimage{max size={0.9\linewidth}{0.9\paperheight}}{output_7_0.png}
    \end{center}
    { \hspace*{\fill} \\}
    
    \begin{tcolorbox}[breakable, size=fbox, boxrule=1pt, pad at break*=1mm,colback=cellbackground, colframe=cellborder]
\prompt{In}{incolor}{10}{\boxspacing}
\begin{Verbatim}[commandchars=\\\{\}]
\PY{c+c1}{\PYZsh{} the bigger the size of the kernel, the more blury and all\PYZhy{}grey the image gets}
\PY{n}{fig}\PY{p}{,} \PY{n}{axes} \PY{o}{=} \PY{n}{plt}\PY{o}{.}\PY{n}{subplots}\PY{p}{(}\PY{l+m+mi}{2}\PY{p}{,} \PY{l+m+mi}{2}\PY{p}{)}

\PY{k}{for} \PY{n}{i} \PY{o+ow}{in} \PY{n+nb}{range}\PY{p}{(}\PY{l+m+mi}{4}\PY{p}{)}\PY{p}{:}
    \PY{n}{con\PYZus{}matr} \PY{o}{=} \PY{n}{gaus\PYZus{}matr}\PY{p}{(}\PY{l+m+mi}{8}\PY{p}{,} \PY{l+m+mi}{2}\PY{o}{*}\PY{o}{*}\PY{n}{i}\PY{p}{)}
    \PY{n}{fglenna} \PY{o}{=} \PY{n}{convolution}\PY{p}{(}\PY{n}{l}\PY{p}{,} \PY{n}{con\PYZus{}matr}\PY{p}{)}
    \PY{n}{ax} \PY{o}{=} \PY{n}{axes}\PY{p}{[}\PY{n}{i}\PY{o}{/}\PY{o}{/}\PY{l+m+mi}{2}\PY{p}{,} \PY{n}{i}\PY{o}{\PYZpc{}}\PY{k}{2}]    
    \PY{n}{ax}\PY{o}{.}\PY{n}{imshow}\PY{p}{(}\PY{n}{fglenna}\PY{p}{,}\PY{n}{cmap}\PY{o}{=}\PY{l+s+s1}{\PYZsq{}}\PY{l+s+s1}{Greys\PYZus{}r}\PY{l+s+s1}{\PYZsq{}}\PY{p}{,}\PY{n}{interpolation}\PY{o}{=}\PY{l+s+s1}{\PYZsq{}}\PY{l+s+s1}{none}\PY{l+s+s1}{\PYZsq{}}\PY{p}{)}
    \PY{n}{ax}\PY{o}{.}\PY{n}{set\PYZus{}title}\PY{p}{(}\PY{l+s+s2}{\PYZdq{}}\PY{l+s+s2}{Sigma = }\PY{l+s+si}{\PYZob{}\PYZcb{}}\PY{l+s+s2}{\PYZdq{}}\PY{o}{.}\PY{n}{format}\PY{p}{(}\PY{l+m+mi}{2}\PY{o}{*}\PY{o}{*}\PY{n}{i}\PY{p}{)}\PY{p}{)}
\PY{n}{fig}\PY{o}{.}\PY{n}{suptitle}\PY{p}{(}\PY{l+s+s1}{\PYZsq{}}\PY{l+s+s1}{Gaussian Filter with different sigma, kernel size = 8}\PY{l+s+s1}{\PYZsq{}}\PY{p}{)}\PY{p}{;}
\end{Verbatim}
\end{tcolorbox}

    \begin{center}
    \adjustimage{max size={0.9\linewidth}{0.9\paperheight}}{output_8_0.png}
    \end{center}
    { \hspace*{\fill} \\}
    
    \hypertarget{second-attempt-with-skimage-functions}{%
\subsection{Second attempt with skimage
functions}\label{second-attempt-with-skimage-functions}}

    \hypertarget{influence-of-sigma}{%
\subsubsection{Influence of Sigma}\label{influence-of-sigma}}

    In our second attempt, we used the skimage functions to see how a proper
implementation of the gaussian function looks like. Clearly, the
professional packages creates a very advanced smoothing while preserving
all the colors. Also here, we can see how the amount of blur gets bigger
as we increase our sigma due to that more and more datapoints are taken
into the new image datapoint.

    \begin{tcolorbox}[breakable, size=fbox, boxrule=1pt, pad at break*=1mm,colback=cellbackground, colframe=cellborder]
\prompt{In}{incolor}{17}{\boxspacing}
\begin{Verbatim}[commandchars=\\\{\}]
\PY{n}{fig}\PY{p}{,} \PY{n}{axes} \PY{o}{=} \PY{n}{plt}\PY{o}{.}\PY{n}{subplots}\PY{p}{(}\PY{l+m+mi}{2}\PY{p}{,} \PY{l+m+mi}{2}\PY{p}{)}

\PY{k}{for} \PY{n}{i} \PY{o+ow}{in} \PY{n+nb}{range}\PY{p}{(}\PY{l+m+mi}{4}\PY{p}{)}\PY{p}{:}
    \PY{n}{fglenna} \PY{o}{=} \PY{n}{gaussian}\PY{p}{(}\PY{n}{l}\PY{p}{,} \PY{n}{sigma}\PY{o}{=}\PY{l+m+mi}{2}\PY{o}{*}\PY{o}{*}\PY{p}{(}\PY{n}{i}\PY{p}{)}\PY{p}{)}
    \PY{n}{ax} \PY{o}{=} \PY{n}{axes}\PY{p}{[}\PY{n}{i}\PY{o}{/}\PY{o}{/}\PY{l+m+mi}{2}\PY{p}{,} \PY{n}{i}\PY{o}{\PYZpc{}}\PY{k}{2}]    
    \PY{n}{ax}\PY{o}{.}\PY{n}{imshow}\PY{p}{(}\PY{n}{fglenna}\PY{p}{,}\PY{n}{cmap}\PY{o}{=}\PY{l+s+s1}{\PYZsq{}}\PY{l+s+s1}{Greys\PYZus{}r}\PY{l+s+s1}{\PYZsq{}}\PY{p}{,}\PY{n}{interpolation}\PY{o}{=}\PY{l+s+s1}{\PYZsq{}}\PY{l+s+s1}{none}\PY{l+s+s1}{\PYZsq{}}\PY{p}{)}
    \PY{n}{ax}\PY{o}{.}\PY{n}{set\PYZus{}title}\PY{p}{(}\PY{l+s+s2}{\PYZdq{}}\PY{l+s+s2}{Sigma = }\PY{l+s+si}{\PYZob{}\PYZcb{}}\PY{l+s+s2}{\PYZdq{}}\PY{o}{.}\PY{n}{format}\PY{p}{(}\PY{l+m+mi}{2}\PY{o}{*}\PY{o}{*}\PY{p}{(}\PY{n}{i}\PY{p}{)}\PY{p}{)}\PY{p}{)}
\PY{n}{fig}\PY{o}{.}\PY{n}{suptitle}\PY{p}{(}\PY{l+s+s1}{\PYZsq{}}\PY{l+s+s1}{Gaussian}\PY{l+s+s1}{\PYZsq{}}\PY{p}{)}\PY{p}{;}
\end{Verbatim}
\end{tcolorbox}

    \begin{center}
    \adjustimage{max size={0.9\linewidth}{0.9\paperheight}}{output_12_0.png}
    \end{center}
    { \hspace*{\fill} \\}
    
    \hypertarget{size-of-image}{%
\subsubsection{Size of image}\label{size-of-image}}

    Another interesting feature of the filtering is the size of the initial
image. The sigma (and also the kernel size) are always to be computed in
dependence of the shape of the image. Below, we resclaed our initial
image by the factor for resulting in an image of shape (128, 128)
instead of (512, 512) and applied the same gaussian filter with the same
different sigmas. We can see that the effect of the filtering is much
more extreme (to be specific: 4 times as extreme) as the function now
convolutes the same amount of pixels. But with a lower resolution this
is a way bigger part of the total image.

    \begin{tcolorbox}[breakable, size=fbox, boxrule=1pt, pad at break*=1mm,colback=cellbackground, colframe=cellborder]
\prompt{In}{incolor}{18}{\boxspacing}
\begin{Verbatim}[commandchars=\\\{\}]
\PY{n}{low\PYZus{}res} \PY{o}{=} \PY{n}{cv2}\PY{o}{.}\PY{n}{resize}\PY{p}{(}\PY{n}{l}\PY{p}{,} \PY{n}{dsize}\PY{o}{=}\PY{p}{(}\PY{l+m+mi}{128}\PY{p}{,} \PY{l+m+mi}{128}\PY{p}{)}\PY{p}{,} \PY{n}{interpolation}\PY{o}{=}\PY{k+kc}{None}\PY{p}{)}
\PY{n}{fig}\PY{p}{,} \PY{n}{axes} \PY{o}{=} \PY{n}{plt}\PY{o}{.}\PY{n}{subplots}\PY{p}{(}\PY{l+m+mi}{2}\PY{p}{,} \PY{l+m+mi}{2}\PY{p}{)}

\PY{k}{for} \PY{n}{i} \PY{o+ow}{in} \PY{n+nb}{range}\PY{p}{(}\PY{l+m+mi}{4}\PY{p}{)}\PY{p}{:}
    \PY{n}{fglenna} \PY{o}{=} \PY{n}{gaussian}\PY{p}{(}\PY{n}{low\PYZus{}res}\PY{p}{,} \PY{n}{sigma}\PY{o}{=}\PY{l+m+mi}{2}\PY{o}{*}\PY{o}{*}\PY{p}{(}\PY{n}{i}\PY{p}{)}\PY{p}{)}
    \PY{n}{ax} \PY{o}{=} \PY{n}{axes}\PY{p}{[}\PY{n}{i}\PY{o}{/}\PY{o}{/}\PY{l+m+mi}{2}\PY{p}{,} \PY{n}{i}\PY{o}{\PYZpc{}}\PY{k}{2}]    
    \PY{n}{ax}\PY{o}{.}\PY{n}{imshow}\PY{p}{(}\PY{n}{fglenna}\PY{p}{,}\PY{n}{cmap}\PY{o}{=}\PY{l+s+s1}{\PYZsq{}}\PY{l+s+s1}{Greys\PYZus{}r}\PY{l+s+s1}{\PYZsq{}}\PY{p}{,}\PY{n}{interpolation}\PY{o}{=}\PY{l+s+s1}{\PYZsq{}}\PY{l+s+s1}{none}\PY{l+s+s1}{\PYZsq{}}\PY{p}{)}
    \PY{n}{ax}\PY{o}{.}\PY{n}{set\PYZus{}title}\PY{p}{(}\PY{l+s+s2}{\PYZdq{}}\PY{l+s+s2}{Sigma = }\PY{l+s+si}{\PYZob{}\PYZcb{}}\PY{l+s+s2}{\PYZdq{}}\PY{o}{.}\PY{n}{format}\PY{p}{(}\PY{l+m+mi}{2}\PY{o}{*}\PY{o}{*}\PY{p}{(}\PY{n}{i}\PY{p}{)}\PY{p}{)}\PY{p}{)}
\PY{n}{fig}\PY{o}{.}\PY{n}{suptitle}\PY{p}{(}\PY{l+s+s1}{\PYZsq{}}\PY{l+s+s1}{Gaussian}\PY{l+s+s1}{\PYZsq{}}\PY{p}{)}\PY{p}{;}
\end{Verbatim}
\end{tcolorbox}

    \begin{center}
    \adjustimage{max size={0.9\linewidth}{0.9\paperheight}}{output_15_0.png}
    \end{center}
    { \hspace*{\fill} \\}
    
    \hypertarget{gaussian-magnitude-filtering}{%
\section{Gaussian magnitude
filtering}\label{gaussian-magnitude-filtering}}

    For the Gaussian Magnitude filtering we applied the methods of skiamge
again. As the above researched Gaussian filtering, we can see a
difference in the blur as bigger as sigma gets. In contrast to the
Gaussian filtering, we see here the edges in the picture. That is, the
derivative of the image values, so the local changes.

    \begin{tcolorbox}[breakable, size=fbox, boxrule=1pt, pad at break*=1mm,colback=cellbackground, colframe=cellborder]
\prompt{In}{incolor}{11}{\boxspacing}
\begin{Verbatim}[commandchars=\\\{\}]
\PY{n}{fig}\PY{p}{,} \PY{n}{axes} \PY{o}{=} \PY{n}{plt}\PY{o}{.}\PY{n}{subplots}\PY{p}{(}\PY{l+m+mi}{2}\PY{p}{,} \PY{l+m+mi}{2}\PY{p}{)}

\PY{k}{for} \PY{n}{i} \PY{o+ow}{in} \PY{n+nb}{range}\PY{p}{(}\PY{l+m+mi}{4}\PY{p}{)}\PY{p}{:}
    \PY{n}{fglenna} \PY{o}{=} \PY{n}{gaussian\PYZus{}gradient\PYZus{}magnitude}\PY{p}{(}\PY{n}{l}\PY{p}{,} \PY{n}{sigma}\PY{o}{=}\PY{l+m+mi}{2}\PY{o}{*}\PY{o}{*}\PY{p}{(}\PY{n}{i}\PY{p}{)}\PY{p}{)}
    \PY{n}{ax} \PY{o}{=} \PY{n}{axes}\PY{p}{[}\PY{n}{i}\PY{o}{/}\PY{o}{/}\PY{l+m+mi}{2}\PY{p}{,} \PY{n}{i}\PY{o}{\PYZpc{}}\PY{k}{2}]    
    \PY{n}{ax}\PY{o}{.}\PY{n}{imshow}\PY{p}{(}\PY{n}{fglenna}\PY{p}{,}\PY{n}{cmap}\PY{o}{=}\PY{l+s+s1}{\PYZsq{}}\PY{l+s+s1}{Greys\PYZus{}r}\PY{l+s+s1}{\PYZsq{}}\PY{p}{,}\PY{n}{interpolation}\PY{o}{=}\PY{l+s+s1}{\PYZsq{}}\PY{l+s+s1}{none}\PY{l+s+s1}{\PYZsq{}}\PY{p}{)}
    \PY{n}{ax}\PY{o}{.}\PY{n}{set\PYZus{}title}\PY{p}{(}\PY{l+s+s2}{\PYZdq{}}\PY{l+s+s2}{Sigma = }\PY{l+s+si}{\PYZob{}\PYZcb{}}\PY{l+s+s2}{\PYZdq{}}\PY{o}{.}\PY{n}{format}\PY{p}{(}\PY{l+m+mi}{2}\PY{o}{*}\PY{o}{*}\PY{p}{(}\PY{n}{i}\PY{p}{)}\PY{p}{)}\PY{p}{)}
\PY{n}{fig}\PY{o}{.}\PY{n}{suptitle}\PY{p}{(}\PY{l+s+s1}{\PYZsq{}}\PY{l+s+s1}{Gaussian Magnitude}\PY{l+s+s1}{\PYZsq{}}\PY{p}{)}\PY{p}{;}
\end{Verbatim}
\end{tcolorbox}

    \begin{center}
    \adjustimage{max size={0.9\linewidth}{0.9\paperheight}}{output_18_0.png}
    \end{center}
    { \hspace*{\fill} \\}
    
    Our attempt of applying first the Gaussian filtering and afterwards the
Gaussian Gradient Magnitude doesn't show any different results, as you
can see below. One reason for that could be, that the skiamge function
already includes a Gaussian smoothing in its GGM function.

    \begin{tcolorbox}[breakable, size=fbox, boxrule=1pt, pad at break*=1mm,colback=cellbackground, colframe=cellborder]
\prompt{In}{incolor}{12}{\boxspacing}
\begin{Verbatim}[commandchars=\\\{\}]
\PY{n}{fig}\PY{p}{,} \PY{n}{axes} \PY{o}{=} \PY{n}{plt}\PY{o}{.}\PY{n}{subplots}\PY{p}{(}\PY{l+m+mi}{2}\PY{p}{,} \PY{l+m+mi}{2}\PY{p}{)}

\PY{k}{for} \PY{n}{i} \PY{o+ow}{in} \PY{n+nb}{range}\PY{p}{(}\PY{l+m+mi}{4}\PY{p}{)}\PY{p}{:}
    \PY{n}{fglenna} \PY{o}{=} \PY{n}{gaussian\PYZus{}gradient\PYZus{}magnitude}\PY{p}{(}\PY{n}{gaussian}\PY{p}{(}\PY{n}{l}\PY{p}{,} \PY{n}{sigma}\PY{o}{=}\PY{l+m+mi}{2}\PY{o}{*}\PY{o}{*}\PY{p}{(}\PY{n}{i}\PY{p}{)}\PY{p}{)}\PY{p}{,} \PY{n}{sigma}\PY{o}{=}\PY{l+m+mi}{2}\PY{o}{*}\PY{o}{*}\PY{p}{(}\PY{n}{i}\PY{p}{)}\PY{p}{)}
    \PY{n}{ax} \PY{o}{=} \PY{n}{axes}\PY{p}{[}\PY{n}{i}\PY{o}{/}\PY{o}{/}\PY{l+m+mi}{2}\PY{p}{,} \PY{n}{i}\PY{o}{\PYZpc{}}\PY{k}{2}]    
    \PY{n}{ax}\PY{o}{.}\PY{n}{imshow}\PY{p}{(}\PY{n}{fglenna}\PY{p}{,}\PY{n}{cmap}\PY{o}{=}\PY{l+s+s1}{\PYZsq{}}\PY{l+s+s1}{Greys\PYZus{}r}\PY{l+s+s1}{\PYZsq{}}\PY{p}{,}\PY{n}{interpolation}\PY{o}{=}\PY{l+s+s1}{\PYZsq{}}\PY{l+s+s1}{none}\PY{l+s+s1}{\PYZsq{}}\PY{p}{)}
    \PY{n}{ax}\PY{o}{.}\PY{n}{set\PYZus{}title}\PY{p}{(}\PY{l+s+s2}{\PYZdq{}}\PY{l+s+s2}{Sigma = }\PY{l+s+si}{\PYZob{}\PYZcb{}}\PY{l+s+s2}{\PYZdq{}}\PY{o}{.}\PY{n}{format}\PY{p}{(}\PY{l+m+mi}{2}\PY{o}{*}\PY{o}{*}\PY{p}{(}\PY{n}{i}\PY{p}{)}\PY{p}{)}\PY{p}{)}
\PY{n}{fig}\PY{o}{.}\PY{n}{suptitle}\PY{p}{(}\PY{l+s+s1}{\PYZsq{}}\PY{l+s+s1}{Gaussian Magnitude of Gaussian}\PY{l+s+s1}{\PYZsq{}}\PY{p}{)}\PY{p}{;}
\end{Verbatim}
\end{tcolorbox}

    \begin{center}
    \adjustimage{max size={0.9\linewidth}{0.9\paperheight}}{output_20_0.png}
    \end{center}
    { \hspace*{\fill} \\}
    
    However, if we compute the GGM of the GGM, we can see a result, that
shows the derivative of the derivative. So this means, the black parts
sourrounded by white parts of the images below show where in the
preceding computation of the GGM there was an edge detected or in other
words, where the values for the image changed.

    \begin{tcolorbox}[breakable, size=fbox, boxrule=1pt, pad at break*=1mm,colback=cellbackground, colframe=cellborder]
\prompt{In}{incolor}{13}{\boxspacing}
\begin{Verbatim}[commandchars=\\\{\}]
\PY{n}{fig}\PY{p}{,} \PY{n}{axes} \PY{o}{=} \PY{n}{plt}\PY{o}{.}\PY{n}{subplots}\PY{p}{(}\PY{l+m+mi}{2}\PY{p}{,} \PY{l+m+mi}{2}\PY{p}{)}

\PY{k}{for} \PY{n}{i} \PY{o+ow}{in} \PY{n+nb}{range}\PY{p}{(}\PY{l+m+mi}{4}\PY{p}{)}\PY{p}{:}
    \PY{n}{fglenna} \PY{o}{=} \PY{n}{gaussian\PYZus{}gradient\PYZus{}magnitude}\PY{p}{(}\PY{n}{gaussian\PYZus{}gradient\PYZus{}magnitude}\PY{p}{(}\PY{n}{gaussian}\PY{p}{(}\PY{n}{l}\PY{p}{,} \PY{n}{sigma}\PY{o}{=}\PY{l+m+mi}{2}\PY{o}{*}\PY{o}{*}\PY{p}{(}\PY{n}{i}\PY{p}{)}\PY{p}{)}\PY{p}{,} \PY{n}{sigma}\PY{o}{=}\PY{l+m+mi}{2}\PY{o}{*}\PY{o}{*}\PY{p}{(}\PY{n}{i}\PY{p}{)}\PY{p}{)}\PY{p}{,} \PY{n}{sigma}\PY{o}{=}\PY{l+m+mi}{2}\PY{o}{*}\PY{o}{*}\PY{p}{(}\PY{n}{i}\PY{p}{)}\PY{p}{)}
    \PY{n}{ax} \PY{o}{=} \PY{n}{axes}\PY{p}{[}\PY{n}{i}\PY{o}{/}\PY{o}{/}\PY{l+m+mi}{2}\PY{p}{,} \PY{n}{i}\PY{o}{\PYZpc{}}\PY{k}{2}]    
    \PY{n}{ax}\PY{o}{.}\PY{n}{imshow}\PY{p}{(}\PY{n}{fglenna}\PY{p}{,}\PY{n}{cmap}\PY{o}{=}\PY{l+s+s1}{\PYZsq{}}\PY{l+s+s1}{Greys\PYZus{}r}\PY{l+s+s1}{\PYZsq{}}\PY{p}{,}\PY{n}{interpolation}\PY{o}{=}\PY{l+s+s1}{\PYZsq{}}\PY{l+s+s1}{none}\PY{l+s+s1}{\PYZsq{}}\PY{p}{)}
    \PY{n}{ax}\PY{o}{.}\PY{n}{set\PYZus{}title}\PY{p}{(}\PY{l+s+s2}{\PYZdq{}}\PY{l+s+s2}{Sigma = }\PY{l+s+si}{\PYZob{}\PYZcb{}}\PY{l+s+s2}{\PYZdq{}}\PY{o}{.}\PY{n}{format}\PY{p}{(}\PY{l+m+mi}{2}\PY{o}{*}\PY{o}{*}\PY{p}{(}\PY{n}{i}\PY{p}{)}\PY{p}{)}\PY{p}{)}
\PY{n}{fig}\PY{o}{.}\PY{n}{suptitle}\PY{p}{(}\PY{l+s+s1}{\PYZsq{}}\PY{l+s+s1}{Gaussian Magnitude of Gaussian Magnitude of Gaussian}\PY{l+s+s1}{\PYZsq{}}\PY{p}{)}\PY{p}{;}
\end{Verbatim}
\end{tcolorbox}

    \begin{center}
    \adjustimage{max size={0.9\linewidth}{0.9\paperheight}}{output_22_0.png}
    \end{center}
    { \hspace*{\fill} \\}
    
    The same is true for when we apply the function three times, where now
every edge splits up again in two edges resulting in an even more
psychadelic picture.

    \begin{tcolorbox}[breakable, size=fbox, boxrule=1pt, pad at break*=1mm,colback=cellbackground, colframe=cellborder]
\prompt{In}{incolor}{14}{\boxspacing}
\begin{Verbatim}[commandchars=\\\{\}]
\PY{n}{fig}\PY{p}{,} \PY{n}{axes} \PY{o}{=} \PY{n}{plt}\PY{o}{.}\PY{n}{subplots}\PY{p}{(}\PY{l+m+mi}{2}\PY{p}{,} \PY{l+m+mi}{2}\PY{p}{)}

\PY{k}{for} \PY{n}{i} \PY{o+ow}{in} \PY{n+nb}{range}\PY{p}{(}\PY{l+m+mi}{4}\PY{p}{)}\PY{p}{:}
    \PY{n}{fglenna} \PY{o}{=} \PY{n}{gaussian\PYZus{}gradient\PYZus{}magnitude}\PY{p}{(}\PY{n}{gaussian\PYZus{}gradient\PYZus{}magnitude}\PY{p}{(}\PY{n}{gaussian\PYZus{}gradient\PYZus{}magnitude}\PY{p}{(}\PY{n}{gaussian}\PY{p}{(}\PY{n}{l}\PY{p}{,} \PY{n}{sigma}\PY{o}{=}\PY{l+m+mi}{2}\PY{o}{*}\PY{o}{*}\PY{p}{(}\PY{n}{i}\PY{p}{)}\PY{p}{)}\PY{p}{,} \PY{n}{sigma}\PY{o}{=}\PY{l+m+mi}{2}\PY{o}{*}\PY{o}{*}\PY{p}{(}\PY{n}{i}\PY{p}{)}\PY{p}{,} \PY{n}{mode}\PY{o}{=}\PY{l+s+s1}{\PYZsq{}}\PY{l+s+s1}{reflect}\PY{l+s+s1}{\PYZsq{}}\PY{p}{)}\PY{p}{,} \PY{n}{sigma}\PY{o}{=}\PY{l+m+mi}{2}\PY{o}{*}\PY{o}{*}\PY{p}{(}\PY{n}{i}\PY{p}{)}\PY{p}{,} \PY{n}{mode}\PY{o}{=}\PY{l+s+s1}{\PYZsq{}}\PY{l+s+s1}{reflect}\PY{l+s+s1}{\PYZsq{}}\PY{p}{)}\PY{p}{,} \PY{n}{sigma}\PY{o}{=}\PY{l+m+mi}{2}\PY{o}{*}\PY{o}{*}\PY{p}{(}\PY{n}{i}\PY{p}{)}\PY{p}{,} \PY{n}{mode}\PY{o}{=}\PY{l+s+s1}{\PYZsq{}}\PY{l+s+s1}{reflect}\PY{l+s+s1}{\PYZsq{}}\PY{p}{)}
    \PY{n}{ax} \PY{o}{=} \PY{n}{axes}\PY{p}{[}\PY{n}{i}\PY{o}{/}\PY{o}{/}\PY{l+m+mi}{2}\PY{p}{,} \PY{n}{i}\PY{o}{\PYZpc{}}\PY{k}{2}]    
    \PY{n}{ax}\PY{o}{.}\PY{n}{imshow}\PY{p}{(}\PY{n}{fglenna}\PY{p}{,}\PY{n}{cmap}\PY{o}{=}\PY{l+s+s1}{\PYZsq{}}\PY{l+s+s1}{Greys\PYZus{}r}\PY{l+s+s1}{\PYZsq{}}\PY{p}{,}\PY{n}{interpolation}\PY{o}{=}\PY{l+s+s1}{\PYZsq{}}\PY{l+s+s1}{none}\PY{l+s+s1}{\PYZsq{}}\PY{p}{)}
    \PY{n}{ax}\PY{o}{.}\PY{n}{set\PYZus{}title}\PY{p}{(}\PY{l+s+s2}{\PYZdq{}}\PY{l+s+s2}{Sigma = }\PY{l+s+si}{\PYZob{}\PYZcb{}}\PY{l+s+s2}{\PYZdq{}}\PY{o}{.}\PY{n}{format}\PY{p}{(}\PY{l+m+mi}{2}\PY{o}{*}\PY{o}{*}\PY{p}{(}\PY{n}{i}\PY{p}{)}\PY{p}{)}\PY{p}{)}
\PY{n}{fig}\PY{o}{.}\PY{n}{suptitle}\PY{p}{(}\PY{l+s+s1}{\PYZsq{}}\PY{l+s+s1}{Gaussian Magnitude of Gaussian Magnitude of Gaussian Magnitude of Gaussian}\PY{l+s+s1}{\PYZsq{}}\PY{p}{)}\PY{p}{;}
\end{Verbatim}
\end{tcolorbox}

    \begin{center}
    \adjustimage{max size={0.9\linewidth}{0.9\paperheight}}{output_24_0.png}
    \end{center}
    { \hspace*{\fill} \\}
    

    % Add a bibliography block to the postdoc
    
    
    
\end{document}
